\documentclass{article}
\usepackage{amsmath}
\usepackage{amsthm}
\usepackage{amssymb}
\usepackage{algorithm}
\usepackage{graphicx}
\usepackage[noend]{algpseudocode}
\usepackage{url}
\usepackage{listings}

\newlength\tindent
\setlength{\tindent}{\parindent}
\setlength{\parindent}{0pt}
\renewcommand{\indent}{\hspace*{\tindent}}

\title{COMP 312 Assignment 6}
\author{Daniel Braithwaite}

\begin{document}
	\pagenumbering{gobble}
	\maketitle
	\newpage
  	\pagenumbering{arabic}
  	
  	\section{Python}
  		\subsection{Program A}
			\subsubsection{Code}
				\lstinputlisting[language=Python]{Programs/coxian.py}
			\subsubsection{Output}
				$E[t]$: (0.75741101633238683, 0.78099765992129067)\\
$E[t^2]$: (0.92233300526267215, 0.98510362267500762)

			The theoretical values are within the 95\% confidence interval which means we can conclude that $E[t]$ and $E[t^2]$ are equal to the theoretical values

  		
  		
  		\subsection{Program B}
  			\subsubsection{Code}
				\lstinputlisting[language=Python]{Programs/mg1-coxian.py}
			\subsubsection{Output}
				Mean Y: -0.00233069473309\\
Conf Y: (-0.011739640704165668, 0.0070782512379870006)\\

				We want Y to be 0 as this would show $w_q = \frac{\lambda E[t^2]}{2(1-\rho)}$. We see that 0 is within the confidence interval of the average Y. Not only this but our mean Y is small enough for us to conclude that $w_q = \frac{\lambda E[t^2]}{2(1-\rho)}$


\end{document}