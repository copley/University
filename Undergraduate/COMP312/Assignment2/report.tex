\documentclass{article}
\usepackage{amsmath}
\usepackage{amsthm}
\usepackage{amssymb}
\usepackage{algorithm}
\usepackage{graphicx}
\usepackage[noend]{algpseudocode}
\usepackage{url}
\usepackage{listings}

\newlength\tindent
\setlength{\tindent}{\parindent}
\setlength{\parindent}{0pt}
\renewcommand{\indent}{\hspace*{\tindent}}

\newtheorem{thm}{Theorem}
\newtheorem{cor}{Corollary}[thm]
\newtheorem{lemma}{Lemma}[thm]

\title{COMP 312 Assignment 2}
\author{Daniel Braithwaite}

\begin{document}
	\pagenumbering{gobble}
	\maketitle
	\newpage
  	\pagenumbering{arabic}
  	
  	\section{Python}
  		\subsection{Dice Problem}
  			\subsubsection{Part A}
  				\lstinputlisting[language=Python]{Programs/01.py}
  				
  				The output of this program is $0.491206$ so yes the probability of the event occurring is less than $\frac{1}{2}$. The value of n was very large so we can be confident in this finding.
  				
  			\subsubsection{Part B}
  				\lstinputlisting[language=Python]{Programs/02.py}
  				
  				The modified code calculates the probability of the event occurring with a different number of throws. Over the range 24 to 26 it outputs the the following $\{24: 0.491443, 25: 0.506207, 26: 0.518883\}$
  				
  				From this we see that it requires 25 throws for the probability to be $ > \frac{1}{2}$ but as close to $\frac{1}{2}$ as possible
  				
  				
  		\subsection{Optimized Table Look-up}
  			\lstinputlisting[language=Python]{Programs/03.py}
			
			The output from running this code is
			\newline
			
			\begin{center}
				\begin{tabular}{ c c c}
				Part I & Part II & Part III\\
				3.750354 & 3.750114 & 3.750704\\
				4.750354 & 2.249886 & 2.064436\\
				
				\end{tabular}
			\end{center}						
			
			As we can see from this all results are mostly the same. We also see that option 2 and 3 take about half as many steps as option 1. However option 3 is just slightly faster
			
	
	\section{Queues}			  			
		\subsection{Problem 3}
			We know that the service times are distributed exponentially. And we assume that all the servers have equal average service times, let this time be $m$. Once one of the other services has finished you can begin being served (you are now 'racing' against the remaining 6 people). As the exponential distribution has the property of being memory-less all the services have the same probability of taking time $m$. This means that $\frac{6}{7}$ probability of finishing before atleast one of the other customers being served.
			
		\subsection{Problem 4}
			$state \gets $ number of active pizza shops\newline
		
			For this problem I used python to compute the steady state vector. The following code is what I used to do this.
			\lstinputlisting[language=Python]{Programs/pizza.py}  			
  			
			To get the average state of the system we take the following sum
			
			\begin{align*}
				\sum_{n = 0}^{32} n\pi_n
			\end{align*}			  			
  			
			for which we get the following result  			
  			
  			\begin{align*}
  				Avg\ number\ of\ pizza\ shops:\ 27.6082698882
  			\end{align*}
  			
			To get the fraction of time spent in state $i$ we know this value is $\pi_i$ so if we want to find the fraction of time with 20 or more restaurants this is
			
			\begin{align}
				\sum_{n = 20}^{32} \pi_n = 0.999741852504
			\end{align}					
  			
  			\begin{align}
  				Fraction\ of\ time\ with\ >\ 20\ restrants:\ 0.999741852504
  			\end{align}
\end{document}