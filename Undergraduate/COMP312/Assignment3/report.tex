\documentclass{article}
\usepackage{amsmath}
\usepackage{amsthm}
\usepackage{amssymb}
\usepackage{algorithm}
\usepackage{graphicx}
\usepackage[noend]{algpseudocode}
\usepackage{url}
\usepackage{listings}

\newlength\tindent
\setlength{\tindent}{\parindent}
\setlength{\parindent}{0pt}
\renewcommand{\indent}{\hspace*{\tindent}}

\newtheorem{thm}{Theorem}
\newtheorem{cor}{Corollary}[thm]
\newtheorem{lemma}{Lemma}[thm]

\title{COMP 312 Assignment 3}
\author{Daniel Braithwaite}

\begin{document}
	\pagenumbering{gobble}
	\maketitle
	\newpage
  	\pagenumbering{arabic}
  	
  	\section{Python}
  		\subsection{Ellipse Problem}
  			\lstinputlisting[language=Python]{Programs/01.py}
  			
  			This program outputs
  			
  			\begin{align}
  				Area:\ 157.079632679\\
				Eccentricity:\ 1.0
  			\end{align}

		\subsection{Chi Square Problem}
			\subsubsection{Part B Solution}
				\lstinputlisting[language=Python]{Programs/02.py}			


				The output for this section is the following table\\
				\begin{center}
					\begin{tabular}{ c c c }
						$k$ & $E(X)$ & $var(X)$\\
						\hline
						1 & 1.0310970666411978 & 2.1809895852678212\\
						2 & 2.0196390351470401 & 4.0906849280367874\\
						3 & 3.0104658433374825 & 6.0929914117158352\\
						4 & 3.9938676109902742 & 8.0085366007510803\\
						5 & 4.9926872174841925 & 10.172262735835909\\
						6 & 5.9559869106853442 & 11.465470122520625\\
						7 & 6.9672714050261986 & 13.834253416945165\\
						8 & 8.0347194125058135 & 16.044484561872114\\
					\end{tabular}
				\end{center}
							
			\subsubsection{Part C Solution}
				\lstinputlisting[language=Python]{Programs/03.py}
				
				\newpage				
				
				The output from this section for k = 8 is 
				
				K =  8\\
Conf Mean:  (7.998448633149601, 8.0155757558836989)\\
Conf Var:  (15.966449581811148, 16.135066786590585)

				
				
			\subsubsection{Part D Solution}
				The outputs that are given from the program make sense as we know for the chi square distribution
				
				\begin{align}
					E(X) \gets k\\
					var(X) \gets 2k 
				\end{align}			
				
				And we can see that the output is approximately equal to this.		
\end{document}